% Created 2020-10-20 ter 16:24
% Intended LaTeX compiler: pdflatex
\documentclass{article}
\usepackage[utf8]{inputenc}
\usepackage[T1]{fontenc}
\usepackage{graphicx}
\usepackage{grffile}
\usepackage{longtable}
\usepackage{wrapfig}
\usepackage{rotating}
\usepackage[normalem]{ulem}
\usepackage{amsmath}
\usepackage{textcomp}
\usepackage{amssymb}
\usepackage{capt-of}
\usepackage{hyperref}
\usepackage{cm-unicode}
\author{Gustavo}
\date{\today}
\title{}
\hypersetup{
 pdfauthor={Gustavo},
 pdftitle={},
 pdfkeywords={},
 pdfsubject={},
 pdfcreator={Emacs 26.3 (Org mode 9.1.9)}, 
 pdflang={English}}
\begin{document}

\tableofcontents



\section{A Natureza das Funções}
\label{sec:orgd5d9490}

\subsection{Calculando a velocidade das bolinhas}
\label{sec:orged3bc27}
\subsubsection{Problema}
\label{sec:orgb244b3b}
Uma bolinha rola sobre um trilho retilíneo como esquematizado na figura abaixo:

\begin{verbatim}
   -2    -1     0    +1    +2    +3    +4
    |  |  |  |  |  |  |  |  |  |  |  |  | 
    ===================================== = = = =
 🎱_____ᴨ______________________________ _ _ _ _ 
=======/============================== = = = =
      /                    
     /               /‾‾/╷
    /             /‾‾‾‾‾‾/⎸     
   /_____________|‾‾‾‾‾‾|/ 
		  ‾‾‾‾‾‾ 
\end{verbatim}

Quando a bolinha passa pelo botão na marcação -1, ativa um timer de 4 segundos
que dispara a câmera.

\begin{enumerate}
\item Nível 1
\label{sec:org1b463c5}
Qual a velocidade da bolinha, em metros/segundo, se a foto a mostra sobre a marcação +3?

\item Nível 2
\label{sec:org14bfe6f}
Lançando agora três bolinhas em instantes diferentes, temos:
\begin{center}
\begin{tabular}{lr}
Bola & Posição na foto\\
azul & +2\\
branca & +4\\
preta & +5\\
\end{tabular}
\end{center}

Calcule a velocidade de cada bola
\end{enumerate}

\subsubsection{Generalização da resolução}
\label{sec:org1509224}
Para uma bola qualquer que parou na posição \(x\),
\[
v(x) = \frac{1 + x}{4}
\]
\end{document}
