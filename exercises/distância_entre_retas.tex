\documentclass{exam}
\usepackage[utf8]{inputenc}
\usepackage{mathtools}
\usepackage{amsthm}
\usepackage{amsmath}
\usepackage{amssymb}
\usepackage{stmaryrd}
\usepackage{xcolor}
\renewcommand{\baselinestretch}{1.2}
\setlength{\parskip}{1em}

\begin{document}

\begin{center}
\fbox{\fbox{\parbox{5.5in}{\centering
      \textbf{Notação} \\
      Denota-se por $\delta (A, B)$ a distância entre dois objetos geométricos quaisquer.}}}
\end{center}

\vspace{5mm}

\begin{questions}

  \question A distância entre dois conjuntos de pontos A e B é definida como a menor distância entre um par de pontos (a, b) tal que a $\in$ A e b $\in$ B.

  \vspace{2mm}
  
  \begin{parts}
    \part Demonstre que a distância entre uma reta $r$ e uma reta $s$ tal que $r \cap s = A$ é sempre 0. \\
    \textit{Dica: distâncias são sempre números pertencentes aos reais positivos incluindo o zero ou ainda $\delta (A, B) \ge 0, \forall A \ \forall B$}

    {\color{red}
    Por hipótese, isto é, pelo caso que a questão apresenta:
    \begin{equation*}
    \begin{aligned}
      & \exists A \ | \  A \in r, A \in s & \ \wedge \\
      & \delta (A, A) = 0 & \ \wedge \\
      & \delta (A, B) \ge 0 & \ \wedge \\
      & \delta (r, s) = \text{mín}\{ \delta (A, B), \forall A \in r, \forall B \in s \} & \Rightarrow \\
      & \delta (r, s) = 0 & \qed
    \end{aligned}
    \end{equation*}

    Lendo as sentenças acima, teríamos algo como:
    \begin{quote}
      Se existe um ponto $A$ tal que $A$ pertence às duas retas, se a distância entre um ponto e ele mesmo é 0, se qualquer distância entre dois pontos é maior ou igual a 0 e se a distância entre duas retas é a menor distância possível entre dois de seus pontos, então a distância entre $r$ e $s$ é 0. $\qed$
    \end{quote}
    }
    \part Demonstre que a distância entre duas retas paralelas $r // s$ é igual a medida de um segmento de reta $\overline{AB}$, A $\in$ r e B $\in$ s tal que $\overline{AB} \perp r \land \overline{AB} \perp s$

    \begin{subparts}
      \subpart Mostre que a distância entre um ponto $A$ e uma reta $r$ tal que $A \notin r$ é determinada pela medida de um segmento de reta $\overline{AB}$, B $\in$ r tal que $\overline{AB} \perp r$. \\
      \textit{Dica: Utilize o Teorema de Pitágoras}

      {\color{red}
        A demonstração pode ser feita por contradição. Primeiro assumimos que o que queremos provar é falso e, por termos assumido falsidade, chegamos numa contradição. Portanto, a afirmação só pode ser verdadeira (caso contrário, nossa matemática seria inconsistente).

        Demonstração formal:
        \begin{equation*}
        \begin{aligned}
          & \text{Seja } P \in r \mid P \ne B \land \delta (A, P) < \delta (A, B)                 &               \\
          & \Rightarrow \delta ^2 (A, P) < \delta ^2 (A, B)                                       &               \\
          & \Rightarrow \delta ^2(A, P) - \delta ^2(A, B) < 0                                     & \ \text{(I)}  \\
          &                                                                                       &               \\
          & \overline{AB} \perp r \land B, P \in r                                                &               \\
          & \xRightarrow{\text{Pitágoras}} \delta ^2 (A, P) = \delta ^2 (A, B) + \delta ^2 (B, P) &               \\
          & \Rightarrow \delta ^2 (B, P) = \delta ^2 (A, P) - \delta ^2 (A, B)                    & \ \text{(II)} \\
          &                                                                                       &               \\
          & \text{(I)} \land \text{(II)}                                                          &               \\      
          & \Rightarrow \delta ^2 (B, P) < 0                                                      & \ \lightning  \\
          &                                                                                       &               \\
          & \therefore \nexists P \in r \mid P \ne B \land \delta (A, P) < \delta (A, B)          &               \\
          & \Rightarrow \text{mín}\{ \delta (A, P), \forall P \in r\} = \delta (A, B)             &               \\
          & \xRightarrow{\text{Definição}} \delta (A, r) = \delta (A, B), \overline{AB} \perp r   & \ \qed        \\      
        \end{aligned}
        \end{equation*}
        
        Lê-se como:
        \begin{quote}
          Tomando um ponto $P$ em $r$ tal que $P$ não seja $B$ e assumindo que sua distância até $A$ é menor do que a distância de $A$ até $B$, podemos dizer que o quadrado da distância de $A$ até $P$ é menor que o quadrado da distância de $A$ até $B$. Isto implica que a diferença entre o quadrado da distância de $A$ até $P$ e o quadrado da distância de $A$ até $B$ é negativa. Esta é a conclusão (I). \\
          \\
          Se o segmento de $A$ a $B$ é perpendicular à reta $r$ e $B$ e $P$ pertencem a $r$, então, por Pitágoras, o quadrado da distância de $A$ até $P$ é igual a soma do quadrado da distância de $A$ até $B$ com o quadrado da distância de $B$ até $P$. Isto implica que o quadrado da distância de $B$ até $P$ é igual a diferença entre o quadrado da distância de $A$ até $P$ e o quadrado da distância de $A$ até $B$. Esta é a conclusão (II). \\
          \\
          Se concluímos (II), que o quadrado da distância de $B$ até $P$ é igual a diferença entre o quadrado da distância de $A$ até $P$ e o quadrado da distância de $A$ até $B$, e concluímos (I), que diz que essa diferença é negativa, então o quadrado da distância de $B$ até $P$ é negativo. Porém o quadrado de um número real nunca pode ser negativo: chegamos a uma contradição. \\
          \\
          Portanto, não existe um $P$ em $r$ tal que $P$ seja diferente de $B$ e que a distância de $A$ até $P$ seja menor que a distância de $A$ até $B$. Isto implica que a mínima distância entre um ponto em $P$ qualquer em $r$ e o ponto $A$ é a distância entre $A$ e $B$. Isto implica, pela definição dada da distância entre dois conjuntos de pontos, que a distância entre a reta $r$ e o ponto $A$ é a distância entre $A$ e $B$ onde o segmento de $A$ até $B$ é perpendicular à reta $r$. $\qed$  
        \end{quote}
      }

      \subpart Mostre que, por simetria, todos os pontos de uma reta s tal que $s // r$ tem a mesma distância em relação a r. Determine, portanto, $\delta (r, s)$.
               {\color{red}
                 \begin{quote}
                   Como as retas paralelas $r$ e $s$ são infinitas, existe uma simetria de translação entre elas. Isto significa que eu posso deslizar uma delas em relação a outra por qualquer número de unidades e teremos o mesmo arranjo de antes da translação. Portanto, uma propriedade do ponto $A \in r$ em relação à reta $s$ valerá igualmente para todos os outros pontos em $r$. Como demonstrado no item anterior, a distância entre um ponto e uma reta é a medida de um segmento de reta que passa pelo ponto e é perpendicular à reta. Desta forma, a distância entre as retas $r$ e $s$ pode ser definida como a distância de \emph{qualquer ponto} em r até s.
                 \end{quote}
               }
      
    \end{subparts}

    \part Como você calcularia $\delta (r, s)$ se r e s são retas reversas?
          {\color{red} Questão Aberta.}
  \end{parts}
  
\end{questions}
\end{document}
