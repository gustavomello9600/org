\documentclass{exam}
\usepackage[utf8]{inputenc}
\usepackage{amsthm}
\usepackage{amsmath}
<<<<<<< HEAD
\usepackage{xcolor}
=======
>>>>>>> d4fd224e3364a5c5020c95a4ec72da6a035773b2
\renewcommand{\baselinestretch}{1.2}
\setlength{\parskip}{1em}

\begin{document}

\begin{center}
\fbox{\fbox{\parbox{5.5in}{\centering
      \textbf{Notação} \\
      Denota-se por $\delta (A, B)$ a distância entre dois objetos geométricos quaisquer.}}}
\end{center}

\vspace{5mm}

\begin{questions}

  \question A distância entre dois conjuntos de pontos A e B é definida como a menor distância entre um par de pontos (a, b) tal que a $\in$ A e b $\in$ B.

  \vspace{2mm}
  
  \begin{parts}
    \part Demonstre que a distância entre uma reta $r$ e uma reta $s$ tal que $r \cap s = A$ é sempre 0. \\
    \textit{Dica: distâncias são sempre números pertencentes aos reais positivos incluindo o zero ou ainda $\delta (A, B) \ge 0, \forall \ A \forall \ B$}

<<<<<<< HEAD
    {\color{gray}
    Por hipótese, isto é, pelo caso que a questão apresenta:
    \begin{equation*}
    \begin{aligned}
      \exists A | A \in r, A \in s & \wedge \\
      \delta (A, A) = 0 & \wedge \\
      \delta (A, B) \ge 0 & \wedge \\
      \delta (r, s) = \text{mín}(\{ \delta (A, B), \forall A \in r, \forall B \in s \}) & \Rightarrow \\
      \delta (r, s) = 0 & \qed
    \end{aligned}
    \end{equation*}

    Lendo as sentenças acima, teríamos algo como:
    \begin{quote}
      Se existe um ponto $A$ tal que $A$ pertence às duas retas, se a distância entre um ponto e ele mesmo é 0, se qualquer distância entre dois pontos é maior ou igual a 0 e se a distância entre duas retas é a menor distância possível entre dois de seus pontos, então a distância entre $r$ e $s$ é 0. $\qed$
    \end{quote}
    }
=======
    Por hipótese, isto é, pelo caso que a questão apresenta:
    \begin{equation*}
    \begin{aligned}
    \exists A | A \in r, A \in s & \wedge \\
    \delta (A, A) = 0 & \wedge \\
    \delta (A, B) \ge 0 & \wedge \\
    \delta (r, s) = \text{mín}(\{ \delta (A, B), \forall A \in r, \forall B \in s \}) & \Rightarrow \\
    \delta (r, s) = 0 & \qed
    \end{aligned}
    \end{equation*}

>>>>>>> d4fd224e3364a5c5020c95a4ec72da6a035773b2
    \part Demonstre que a distância entre duas retas paralelas $r // s$ é igual a medida de um segmento de reta $\overline{AB}$, A $\in$ r e B $\in$ s tal que $\overline{AB} \perp r \land \overline{AB} \perp s$

    \begin{subparts}
      \subpart Mostre que a distância entre um ponto A e uma reta r tal que A $\notin$ r é determinada pela medida de um segmento de reta $\overline{AB}$, B $\in$ r tal que $\overline{AB} \perp r$. \\
      \textit{Dica: Utilize o Teorema de Pitágoras}

      \subpart Mostre que, por simetria, todos os pontos de uma reta s tal que $s // r$ tem a mesma distância em relação a r. Determine, portanto, $\delta (r, s)$.
    \end{subparts}

    \part Como você calcularia $\delta (r, s)$ se r e s são retas reversas?
  \end{parts}
  
\end{questions}
\end{document}
